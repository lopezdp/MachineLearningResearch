
% Default to the notebook output style

    


% Inherit from the specified cell style.




    
\documentclass[11pt]{article}

    
    
    \usepackage[T1]{fontenc}
    % Nicer default font (+ math font) than Computer Modern for most use cases
    \usepackage{mathpazo}

    % Basic figure setup, for now with no caption control since it's done
    % automatically by Pandoc (which extracts ![](path) syntax from Markdown).
    \usepackage{graphicx}
    % We will generate all images so they have a width \maxwidth. This means
    % that they will get their normal width if they fit onto the page, but
    % are scaled down if they would overflow the margins.
    \makeatletter
    \def\maxwidth{\ifdim\Gin@nat@width>\linewidth\linewidth
    \else\Gin@nat@width\fi}
    \makeatother
    \let\Oldincludegraphics\includegraphics
    % Set max figure width to be 80% of text width, for now hardcoded.
    \renewcommand{\includegraphics}[1]{\Oldincludegraphics[width=.8\maxwidth]{#1}}
    % Ensure that by default, figures have no caption (until we provide a
    % proper Figure object with a Caption API and a way to capture that
    % in the conversion process - todo).
    \usepackage{caption}
    \DeclareCaptionLabelFormat{nolabel}{}
    \captionsetup{labelformat=nolabel}

    \usepackage{adjustbox} % Used to constrain images to a maximum size 
    \usepackage{xcolor} % Allow colors to be defined
    \usepackage{enumerate} % Needed for markdown enumerations to work
    \usepackage{geometry} % Used to adjust the document margins
    \usepackage{amsmath} % Equations
    \usepackage{amssymb} % Equations
    \usepackage{textcomp} % defines textquotesingle
    % Hack from http://tex.stackexchange.com/a/47451/13684:
    \AtBeginDocument{%
        \def\PYZsq{\textquotesingle}% Upright quotes in Pygmentized code
    }
    \usepackage{upquote} % Upright quotes for verbatim code
    \usepackage{eurosym} % defines \euro
    \usepackage[mathletters]{ucs} % Extended unicode (utf-8) support
    \usepackage[utf8x]{inputenc} % Allow utf-8 characters in the tex document
    \usepackage{fancyvrb} % verbatim replacement that allows latex
    \usepackage{grffile} % extends the file name processing of package graphics 
                         % to support a larger range 
    % The hyperref package gives us a pdf with properly built
    % internal navigation ('pdf bookmarks' for the table of contents,
    % internal cross-reference links, web links for URLs, etc.)
    \usepackage{hyperref}
    \usepackage{longtable} % longtable support required by pandoc >1.10
    \usepackage{booktabs}  % table support for pandoc > 1.12.2
    \usepackage[inline]{enumitem} % IRkernel/repr support (it uses the enumerate* environment)
    \usepackage[normalem]{ulem} % ulem is needed to support strikethroughs (\sout)
                                % normalem makes italics be italics, not underlines
    

    
    
    % Colors for the hyperref package
    \definecolor{urlcolor}{rgb}{0,.145,.698}
    \definecolor{linkcolor}{rgb}{.71,0.21,0.01}
    \definecolor{citecolor}{rgb}{.12,.54,.11}

    % ANSI colors
    \definecolor{ansi-black}{HTML}{3E424D}
    \definecolor{ansi-black-intense}{HTML}{282C36}
    \definecolor{ansi-red}{HTML}{E75C58}
    \definecolor{ansi-red-intense}{HTML}{B22B31}
    \definecolor{ansi-green}{HTML}{00A250}
    \definecolor{ansi-green-intense}{HTML}{007427}
    \definecolor{ansi-yellow}{HTML}{DDB62B}
    \definecolor{ansi-yellow-intense}{HTML}{B27D12}
    \definecolor{ansi-blue}{HTML}{208FFB}
    \definecolor{ansi-blue-intense}{HTML}{0065CA}
    \definecolor{ansi-magenta}{HTML}{D160C4}
    \definecolor{ansi-magenta-intense}{HTML}{A03196}
    \definecolor{ansi-cyan}{HTML}{60C6C8}
    \definecolor{ansi-cyan-intense}{HTML}{258F8F}
    \definecolor{ansi-white}{HTML}{C5C1B4}
    \definecolor{ansi-white-intense}{HTML}{A1A6B2}

    % commands and environments needed by pandoc snippets
    % extracted from the output of `pandoc -s`
    \providecommand{\tightlist}{%
      \setlength{\itemsep}{0pt}\setlength{\parskip}{0pt}}
    \DefineVerbatimEnvironment{Highlighting}{Verbatim}{commandchars=\\\{\}}
    % Add ',fontsize=\small' for more characters per line
    \newenvironment{Shaded}{}{}
    \newcommand{\KeywordTok}[1]{\textcolor[rgb]{0.00,0.44,0.13}{\textbf{{#1}}}}
    \newcommand{\DataTypeTok}[1]{\textcolor[rgb]{0.56,0.13,0.00}{{#1}}}
    \newcommand{\DecValTok}[1]{\textcolor[rgb]{0.25,0.63,0.44}{{#1}}}
    \newcommand{\BaseNTok}[1]{\textcolor[rgb]{0.25,0.63,0.44}{{#1}}}
    \newcommand{\FloatTok}[1]{\textcolor[rgb]{0.25,0.63,0.44}{{#1}}}
    \newcommand{\CharTok}[1]{\textcolor[rgb]{0.25,0.44,0.63}{{#1}}}
    \newcommand{\StringTok}[1]{\textcolor[rgb]{0.25,0.44,0.63}{{#1}}}
    \newcommand{\CommentTok}[1]{\textcolor[rgb]{0.38,0.63,0.69}{\textit{{#1}}}}
    \newcommand{\OtherTok}[1]{\textcolor[rgb]{0.00,0.44,0.13}{{#1}}}
    \newcommand{\AlertTok}[1]{\textcolor[rgb]{1.00,0.00,0.00}{\textbf{{#1}}}}
    \newcommand{\FunctionTok}[1]{\textcolor[rgb]{0.02,0.16,0.49}{{#1}}}
    \newcommand{\RegionMarkerTok}[1]{{#1}}
    \newcommand{\ErrorTok}[1]{\textcolor[rgb]{1.00,0.00,0.00}{\textbf{{#1}}}}
    \newcommand{\NormalTok}[1]{{#1}}
    
    % Additional commands for more recent versions of Pandoc
    \newcommand{\ConstantTok}[1]{\textcolor[rgb]{0.53,0.00,0.00}{{#1}}}
    \newcommand{\SpecialCharTok}[1]{\textcolor[rgb]{0.25,0.44,0.63}{{#1}}}
    \newcommand{\VerbatimStringTok}[1]{\textcolor[rgb]{0.25,0.44,0.63}{{#1}}}
    \newcommand{\SpecialStringTok}[1]{\textcolor[rgb]{0.73,0.40,0.53}{{#1}}}
    \newcommand{\ImportTok}[1]{{#1}}
    \newcommand{\DocumentationTok}[1]{\textcolor[rgb]{0.73,0.13,0.13}{\textit{{#1}}}}
    \newcommand{\AnnotationTok}[1]{\textcolor[rgb]{0.38,0.63,0.69}{\textbf{\textit{{#1}}}}}
    \newcommand{\CommentVarTok}[1]{\textcolor[rgb]{0.38,0.63,0.69}{\textbf{\textit{{#1}}}}}
    \newcommand{\VariableTok}[1]{\textcolor[rgb]{0.10,0.09,0.49}{{#1}}}
    \newcommand{\ControlFlowTok}[1]{\textcolor[rgb]{0.00,0.44,0.13}{\textbf{{#1}}}}
    \newcommand{\OperatorTok}[1]{\textcolor[rgb]{0.40,0.40,0.40}{{#1}}}
    \newcommand{\BuiltInTok}[1]{{#1}}
    \newcommand{\ExtensionTok}[1]{{#1}}
    \newcommand{\PreprocessorTok}[1]{\textcolor[rgb]{0.74,0.48,0.00}{{#1}}}
    \newcommand{\AttributeTok}[1]{\textcolor[rgb]{0.49,0.56,0.16}{{#1}}}
    \newcommand{\InformationTok}[1]{\textcolor[rgb]{0.38,0.63,0.69}{\textbf{\textit{{#1}}}}}
    \newcommand{\WarningTok}[1]{\textcolor[rgb]{0.38,0.63,0.69}{\textbf{\textit{{#1}}}}}
    
    
    % Define a nice break command that doesn't care if a line doesn't already
    % exist.
    \def\br{\hspace*{\fill} \\* }
    % Math Jax compatability definitions
    \def\gt{>}
    \def\lt{<}
    % Document parameters
    \title{ChallengerORingData}
    
    
    

    % Pygments definitions
    
\makeatletter
\def\PY@reset{\let\PY@it=\relax \let\PY@bf=\relax%
    \let\PY@ul=\relax \let\PY@tc=\relax%
    \let\PY@bc=\relax \let\PY@ff=\relax}
\def\PY@tok#1{\csname PY@tok@#1\endcsname}
\def\PY@toks#1+{\ifx\relax#1\empty\else%
    \PY@tok{#1}\expandafter\PY@toks\fi}
\def\PY@do#1{\PY@bc{\PY@tc{\PY@ul{%
    \PY@it{\PY@bf{\PY@ff{#1}}}}}}}
\def\PY#1#2{\PY@reset\PY@toks#1+\relax+\PY@do{#2}}

\expandafter\def\csname PY@tok@w\endcsname{\def\PY@tc##1{\textcolor[rgb]{0.73,0.73,0.73}{##1}}}
\expandafter\def\csname PY@tok@c\endcsname{\let\PY@it=\textit\def\PY@tc##1{\textcolor[rgb]{0.25,0.50,0.50}{##1}}}
\expandafter\def\csname PY@tok@cp\endcsname{\def\PY@tc##1{\textcolor[rgb]{0.74,0.48,0.00}{##1}}}
\expandafter\def\csname PY@tok@k\endcsname{\let\PY@bf=\textbf\def\PY@tc##1{\textcolor[rgb]{0.00,0.50,0.00}{##1}}}
\expandafter\def\csname PY@tok@kp\endcsname{\def\PY@tc##1{\textcolor[rgb]{0.00,0.50,0.00}{##1}}}
\expandafter\def\csname PY@tok@kt\endcsname{\def\PY@tc##1{\textcolor[rgb]{0.69,0.00,0.25}{##1}}}
\expandafter\def\csname PY@tok@o\endcsname{\def\PY@tc##1{\textcolor[rgb]{0.40,0.40,0.40}{##1}}}
\expandafter\def\csname PY@tok@ow\endcsname{\let\PY@bf=\textbf\def\PY@tc##1{\textcolor[rgb]{0.67,0.13,1.00}{##1}}}
\expandafter\def\csname PY@tok@nb\endcsname{\def\PY@tc##1{\textcolor[rgb]{0.00,0.50,0.00}{##1}}}
\expandafter\def\csname PY@tok@nf\endcsname{\def\PY@tc##1{\textcolor[rgb]{0.00,0.00,1.00}{##1}}}
\expandafter\def\csname PY@tok@nc\endcsname{\let\PY@bf=\textbf\def\PY@tc##1{\textcolor[rgb]{0.00,0.00,1.00}{##1}}}
\expandafter\def\csname PY@tok@nn\endcsname{\let\PY@bf=\textbf\def\PY@tc##1{\textcolor[rgb]{0.00,0.00,1.00}{##1}}}
\expandafter\def\csname PY@tok@ne\endcsname{\let\PY@bf=\textbf\def\PY@tc##1{\textcolor[rgb]{0.82,0.25,0.23}{##1}}}
\expandafter\def\csname PY@tok@nv\endcsname{\def\PY@tc##1{\textcolor[rgb]{0.10,0.09,0.49}{##1}}}
\expandafter\def\csname PY@tok@no\endcsname{\def\PY@tc##1{\textcolor[rgb]{0.53,0.00,0.00}{##1}}}
\expandafter\def\csname PY@tok@nl\endcsname{\def\PY@tc##1{\textcolor[rgb]{0.63,0.63,0.00}{##1}}}
\expandafter\def\csname PY@tok@ni\endcsname{\let\PY@bf=\textbf\def\PY@tc##1{\textcolor[rgb]{0.60,0.60,0.60}{##1}}}
\expandafter\def\csname PY@tok@na\endcsname{\def\PY@tc##1{\textcolor[rgb]{0.49,0.56,0.16}{##1}}}
\expandafter\def\csname PY@tok@nt\endcsname{\let\PY@bf=\textbf\def\PY@tc##1{\textcolor[rgb]{0.00,0.50,0.00}{##1}}}
\expandafter\def\csname PY@tok@nd\endcsname{\def\PY@tc##1{\textcolor[rgb]{0.67,0.13,1.00}{##1}}}
\expandafter\def\csname PY@tok@s\endcsname{\def\PY@tc##1{\textcolor[rgb]{0.73,0.13,0.13}{##1}}}
\expandafter\def\csname PY@tok@sd\endcsname{\let\PY@it=\textit\def\PY@tc##1{\textcolor[rgb]{0.73,0.13,0.13}{##1}}}
\expandafter\def\csname PY@tok@si\endcsname{\let\PY@bf=\textbf\def\PY@tc##1{\textcolor[rgb]{0.73,0.40,0.53}{##1}}}
\expandafter\def\csname PY@tok@se\endcsname{\let\PY@bf=\textbf\def\PY@tc##1{\textcolor[rgb]{0.73,0.40,0.13}{##1}}}
\expandafter\def\csname PY@tok@sr\endcsname{\def\PY@tc##1{\textcolor[rgb]{0.73,0.40,0.53}{##1}}}
\expandafter\def\csname PY@tok@ss\endcsname{\def\PY@tc##1{\textcolor[rgb]{0.10,0.09,0.49}{##1}}}
\expandafter\def\csname PY@tok@sx\endcsname{\def\PY@tc##1{\textcolor[rgb]{0.00,0.50,0.00}{##1}}}
\expandafter\def\csname PY@tok@m\endcsname{\def\PY@tc##1{\textcolor[rgb]{0.40,0.40,0.40}{##1}}}
\expandafter\def\csname PY@tok@gh\endcsname{\let\PY@bf=\textbf\def\PY@tc##1{\textcolor[rgb]{0.00,0.00,0.50}{##1}}}
\expandafter\def\csname PY@tok@gu\endcsname{\let\PY@bf=\textbf\def\PY@tc##1{\textcolor[rgb]{0.50,0.00,0.50}{##1}}}
\expandafter\def\csname PY@tok@gd\endcsname{\def\PY@tc##1{\textcolor[rgb]{0.63,0.00,0.00}{##1}}}
\expandafter\def\csname PY@tok@gi\endcsname{\def\PY@tc##1{\textcolor[rgb]{0.00,0.63,0.00}{##1}}}
\expandafter\def\csname PY@tok@gr\endcsname{\def\PY@tc##1{\textcolor[rgb]{1.00,0.00,0.00}{##1}}}
\expandafter\def\csname PY@tok@ge\endcsname{\let\PY@it=\textit}
\expandafter\def\csname PY@tok@gs\endcsname{\let\PY@bf=\textbf}
\expandafter\def\csname PY@tok@gp\endcsname{\let\PY@bf=\textbf\def\PY@tc##1{\textcolor[rgb]{0.00,0.00,0.50}{##1}}}
\expandafter\def\csname PY@tok@go\endcsname{\def\PY@tc##1{\textcolor[rgb]{0.53,0.53,0.53}{##1}}}
\expandafter\def\csname PY@tok@gt\endcsname{\def\PY@tc##1{\textcolor[rgb]{0.00,0.27,0.87}{##1}}}
\expandafter\def\csname PY@tok@err\endcsname{\def\PY@bc##1{\setlength{\fboxsep}{0pt}\fcolorbox[rgb]{1.00,0.00,0.00}{1,1,1}{\strut ##1}}}
\expandafter\def\csname PY@tok@kc\endcsname{\let\PY@bf=\textbf\def\PY@tc##1{\textcolor[rgb]{0.00,0.50,0.00}{##1}}}
\expandafter\def\csname PY@tok@kd\endcsname{\let\PY@bf=\textbf\def\PY@tc##1{\textcolor[rgb]{0.00,0.50,0.00}{##1}}}
\expandafter\def\csname PY@tok@kn\endcsname{\let\PY@bf=\textbf\def\PY@tc##1{\textcolor[rgb]{0.00,0.50,0.00}{##1}}}
\expandafter\def\csname PY@tok@kr\endcsname{\let\PY@bf=\textbf\def\PY@tc##1{\textcolor[rgb]{0.00,0.50,0.00}{##1}}}
\expandafter\def\csname PY@tok@bp\endcsname{\def\PY@tc##1{\textcolor[rgb]{0.00,0.50,0.00}{##1}}}
\expandafter\def\csname PY@tok@fm\endcsname{\def\PY@tc##1{\textcolor[rgb]{0.00,0.00,1.00}{##1}}}
\expandafter\def\csname PY@tok@vc\endcsname{\def\PY@tc##1{\textcolor[rgb]{0.10,0.09,0.49}{##1}}}
\expandafter\def\csname PY@tok@vg\endcsname{\def\PY@tc##1{\textcolor[rgb]{0.10,0.09,0.49}{##1}}}
\expandafter\def\csname PY@tok@vi\endcsname{\def\PY@tc##1{\textcolor[rgb]{0.10,0.09,0.49}{##1}}}
\expandafter\def\csname PY@tok@vm\endcsname{\def\PY@tc##1{\textcolor[rgb]{0.10,0.09,0.49}{##1}}}
\expandafter\def\csname PY@tok@sa\endcsname{\def\PY@tc##1{\textcolor[rgb]{0.73,0.13,0.13}{##1}}}
\expandafter\def\csname PY@tok@sb\endcsname{\def\PY@tc##1{\textcolor[rgb]{0.73,0.13,0.13}{##1}}}
\expandafter\def\csname PY@tok@sc\endcsname{\def\PY@tc##1{\textcolor[rgb]{0.73,0.13,0.13}{##1}}}
\expandafter\def\csname PY@tok@dl\endcsname{\def\PY@tc##1{\textcolor[rgb]{0.73,0.13,0.13}{##1}}}
\expandafter\def\csname PY@tok@s2\endcsname{\def\PY@tc##1{\textcolor[rgb]{0.73,0.13,0.13}{##1}}}
\expandafter\def\csname PY@tok@sh\endcsname{\def\PY@tc##1{\textcolor[rgb]{0.73,0.13,0.13}{##1}}}
\expandafter\def\csname PY@tok@s1\endcsname{\def\PY@tc##1{\textcolor[rgb]{0.73,0.13,0.13}{##1}}}
\expandafter\def\csname PY@tok@mb\endcsname{\def\PY@tc##1{\textcolor[rgb]{0.40,0.40,0.40}{##1}}}
\expandafter\def\csname PY@tok@mf\endcsname{\def\PY@tc##1{\textcolor[rgb]{0.40,0.40,0.40}{##1}}}
\expandafter\def\csname PY@tok@mh\endcsname{\def\PY@tc##1{\textcolor[rgb]{0.40,0.40,0.40}{##1}}}
\expandafter\def\csname PY@tok@mi\endcsname{\def\PY@tc##1{\textcolor[rgb]{0.40,0.40,0.40}{##1}}}
\expandafter\def\csname PY@tok@il\endcsname{\def\PY@tc##1{\textcolor[rgb]{0.40,0.40,0.40}{##1}}}
\expandafter\def\csname PY@tok@mo\endcsname{\def\PY@tc##1{\textcolor[rgb]{0.40,0.40,0.40}{##1}}}
\expandafter\def\csname PY@tok@ch\endcsname{\let\PY@it=\textit\def\PY@tc##1{\textcolor[rgb]{0.25,0.50,0.50}{##1}}}
\expandafter\def\csname PY@tok@cm\endcsname{\let\PY@it=\textit\def\PY@tc##1{\textcolor[rgb]{0.25,0.50,0.50}{##1}}}
\expandafter\def\csname PY@tok@cpf\endcsname{\let\PY@it=\textit\def\PY@tc##1{\textcolor[rgb]{0.25,0.50,0.50}{##1}}}
\expandafter\def\csname PY@tok@c1\endcsname{\let\PY@it=\textit\def\PY@tc##1{\textcolor[rgb]{0.25,0.50,0.50}{##1}}}
\expandafter\def\csname PY@tok@cs\endcsname{\let\PY@it=\textit\def\PY@tc##1{\textcolor[rgb]{0.25,0.50,0.50}{##1}}}

\def\PYZbs{\char`\\}
\def\PYZus{\char`\_}
\def\PYZob{\char`\{}
\def\PYZcb{\char`\}}
\def\PYZca{\char`\^}
\def\PYZam{\char`\&}
\def\PYZlt{\char`\<}
\def\PYZgt{\char`\>}
\def\PYZsh{\char`\#}
\def\PYZpc{\char`\%}
\def\PYZdl{\char`\$}
\def\PYZhy{\char`\-}
\def\PYZsq{\char`\'}
\def\PYZdq{\char`\"}
\def\PYZti{\char`\~}
% for compatibility with earlier versions
\def\PYZat{@}
\def\PYZlb{[}
\def\PYZrb{]}
\makeatother


    % Exact colors from NB
    \definecolor{incolor}{rgb}{0.0, 0.0, 0.5}
    \definecolor{outcolor}{rgb}{0.545, 0.0, 0.0}



    
    % Prevent overflowing lines due to hard-to-break entities
    \sloppy 
    % Setup hyperref package
    \hypersetup{
      breaklinks=true,  % so long urls are correctly broken across lines
      colorlinks=true,
      urlcolor=urlcolor,
      linkcolor=linkcolor,
      citecolor=citecolor,
      }
    % Slightly bigger margins than the latex defaults
    
    \geometry{verbose,tmargin=1in,bmargin=1in,lmargin=1in,rmargin=1in}
    
    

    \begin{document}
    
    
    \maketitle
    
    

    
    \hypertarget{challenger-usa-space-shuttle-o-ring-data-set}{%
\section{Challenger USA Space Shuttle O-Ring Data
Set}\label{challenger-usa-space-shuttle-o-ring-data-set}}

\textbf{\href{https://archive.ics.uci.edu/ml/datasets/Challenger+USA+Space+Shuttle+O-Ring}{Project
Data Set Source}}

\textbf{Abstract \& Task:} predict the number of O-rings that experience
thermal distress on a flight at 31 degrees F given data on the previous
23 shuttle flights.

\begin{longtable}[]{@{}llllll@{}}
\toprule
\endhead
\begin{minipage}[t]{0.26\columnwidth}\raggedright
\textbf{Data Set Characteristics:}\strut
\end{minipage} & \begin{minipage}[t]{0.10\columnwidth}\raggedright
Multivariate\strut
\end{minipage} & \begin{minipage}[t]{0.22\columnwidth}\raggedright
\textbf{Number of Instances:}\strut
\end{minipage} & \begin{minipage}[t]{0.02\columnwidth}\raggedright
23\strut
\end{minipage} & \begin{minipage}[t]{0.15\columnwidth}\raggedright
\textbf{Area:}\strut
\end{minipage} & \begin{minipage}[t]{0.09\columnwidth}\raggedright
Physical\strut
\end{minipage}\tabularnewline
\begin{minipage}[t]{0.26\columnwidth}\raggedright
\textbf{Attribute Characteristics:}\strut
\end{minipage} & \begin{minipage}[t]{0.10\columnwidth}\raggedright
Integer\strut
\end{minipage} & \begin{minipage}[t]{0.22\columnwidth}\raggedright
\textbf{Number of Attributes:}\strut
\end{minipage} & \begin{minipage}[t]{0.02\columnwidth}\raggedright
4\strut
\end{minipage} & \begin{minipage}[t]{0.15\columnwidth}\raggedright
\textbf{Date Donated:}\strut
\end{minipage} & \begin{minipage}[t]{0.09\columnwidth}\raggedright
1993-08-05\strut
\end{minipage}\tabularnewline
\begin{minipage}[t]{0.26\columnwidth}\raggedright
\textbf{Associated Tasks:}\strut
\end{minipage} & \begin{minipage}[t]{0.10\columnwidth}\raggedright
Regression\strut
\end{minipage} & \begin{minipage}[t]{0.22\columnwidth}\raggedright
\textbf{Missing Vales?}\strut
\end{minipage} & \begin{minipage}[t]{0.02\columnwidth}\raggedright
No\strut
\end{minipage} & \begin{minipage}[t]{0.15\columnwidth}\raggedright
\textbf{Number of Hits}\strut
\end{minipage} & \begin{minipage}[t]{0.09\columnwidth}\raggedright
90012\strut
\end{minipage}\tabularnewline
\bottomrule
\end{longtable}

\hypertarget{source}{%
\subsection{Source:}\label{source}}

\textbf{Original Owner:}

\begin{verbatim}
David Draper (draper '@' math.ucla.edu)
University of California, Los Angeles
\end{verbatim}

\textbf{Donor:}

\begin{verbatim}
David Draper (draper '@' math.ucla.edu)
\end{verbatim}

\hypertarget{data-set-information}{%
\subsection{Data Set Information}\label{data-set-information}}

\textbf{There are two databases (both use the same set of 5
attributes):}

\begin{verbatim}
1. Primary o-ring erosion and/or blowby
2. Primary o-ring erosion only
\end{verbatim}

\textbf{\emph{The two databases are identical except for the 2nd
attribute of the 21st instance (confirmed by David Draper on 8/5/93).}}

\textbf{\emph{Edited from (Draper, 1993):}}

The motivation for collecting this database was the explosion of the USA
Space Shuttle Challenger on 28 January, 1986. An investigation ensued
into the reliability of the shuttle's propulsion system. The explosion
was eventually traced to the failure of one of the three field joints on
one of the two solid booster rockets. Each of these six field joints
includes two O-rings, designated as primary and secondary, which fail
when phenomena called erosion and blowby both occur.

The night before the launch a decision had to be made regarding launch
safety. The discussion among engineers and managers leading to this
decision included concern that the probability of failure of the O-rings
depended on the temperature t at launch, which was forecasted to be 31
degrees F. There are strong engineering reasons based on the composition
of O-rings to support the judgment that failure probability may rise
monotonically as temperature drops. One other variable, the pressures at
which safety testing for field join leaks was performed, was available,
but its relevance to the failure process was unclear.

Draper's paper includes a menacing figure graphing the number of field
joints experiencing stress vs.~liftoff temperature for the 23 shuttle
flights previous to the Challenger disaster. No previous liftoff
temperature was under 53 degrees F. Although tremendous extrapolation
must be done from the given data to assess risk at 31 degrees F, it is
obvious even to the layman ``to foresee the unacceptably high risk
created by launching at 31 degrees F.'' For more information, see Draper
(1993) or the other previous analyses.

\textbf{\emph{The task is to predict the number of O-rings that will
experience thermal distress for a given flight when the launch
temperature is below freezing.}}

\hypertarget{attribute-information}{%
\subsection{Attribute Information:}\label{attribute-information}}

\begin{verbatim}
1. Number of O-rings at risk on a given flight
2. Number experiencing thermal distress
3. Launch temperature (degrees F)
4. Leak-check pressure (psi)
5. Temporal order of flight
\end{verbatim}

\hypertarget{relevant-papers}{%
\subsection{Relevant Papers:}\label{relevant-papers}}

\begin{verbatim}
Draper,D. (1993). Assessment and propagation of model uncertainty. In Proceedings of the Fourth International
Workshop on Artificial Intelligence and Statistics (pp. 497--509). Ft. Lauderdale, FL: Unpublished.

Dalal,S.R., Fowlkes,E.B., & Hoadley,B. (1989). Risk analysis of the space shuttle: pre-Challenger prediction of
failure. Journal of the American Statisticians Association, 84, 945--957.

Lavine,M. (1991). Problems in extrapolation illustrated with space shuttle O-ring data. Journal of the American
Statisticians Association, 86, 919--922.

Martz H.F., & Zimmer, W.J. (1992). The risk of catastrophic failure of the solid rocket boosters on the space
shuttle. American Statistics, 46, 42--47.
\end{verbatim}

\hypertarget{data-set}{%
\subsection{Data Set:}\label{data-set}}

    \begin{Verbatim}[commandchars=\\\{\}]
{\color{incolor}In [{\color{incolor}5}]:} \PY{k+kn}{import} \PY{n+nn}{numpy}
        \PY{k+kn}{import} \PY{n+nn}{pandas}
        \PY{k+kn}{import} \PY{n+nn}{seaborn} \PY{k}{as} \PY{n+nn}{sb}
        \PY{k+kn}{import} \PY{n+nn}{matplotlib}\PY{n+nn}{.}\PY{n+nn}{pyplot} \PY{k}{as} \PY{n+nn}{plt}
        \PY{k+kn}{from} \PY{n+nn}{sklearn}\PY{n+nn}{.}\PY{n+nn}{linear\PYZus{}model} \PY{k}{import} \PY{n}{LinearRegression}
        
        \PY{n}{data} \PY{o}{=} \PY{n}{pandas}\PY{o}{.}\PY{n}{read\PYZus{}csv}\PY{p}{(}\PY{l+s+s2}{\PYZdq{}}\PY{l+s+s2}{data}\PY{l+s+s2}{\PYZbs{}}\PY{l+s+s2}{o\PYZhy{}ring\PYZhy{}erosion\PYZhy{}onlyData}\PY{l+s+s2}{\PYZdq{}}\PY{p}{,} \PY{n}{header} \PY{o}{=} \PY{k+kc}{None}\PY{p}{)}
        
        \PY{n}{data}\PY{o}{.}\PY{n}{columns} \PY{o}{=} \PY{p}{[}\PY{l+s+s2}{\PYZdq{}}\PY{l+s+s2}{Number of O\PYZhy{}rings at risk on a given flight}\PY{l+s+s2}{\PYZdq{}}\PY{p}{,} \PY{l+s+s2}{\PYZdq{}}\PY{l+s+s2}{Number experiencing thermal distress}\PY{l+s+s2}{\PYZdq{}}\PY{p}{,} \PY{l+s+s2}{\PYZdq{}}\PY{l+s+s2}{Launch temperature (degrees F)}\PY{l+s+s2}{\PYZdq{}}\PY{p}{,} \PY{l+s+s2}{\PYZdq{}}\PY{l+s+s2}{Leak\PYZhy{}check pressure (psi)}\PY{l+s+s2}{\PYZdq{}}\PY{p}{,} \PY{l+s+s2}{\PYZdq{}}\PY{l+s+s2}{Temporal order of flight}\PY{l+s+s2}{\PYZdq{}}\PY{p}{]}
        
        \PY{n}{data}
\end{Verbatim}


\begin{Verbatim}[commandchars=\\\{\}]
{\color{outcolor}Out[{\color{outcolor}5}]:}     Number of O-rings at risk on a given flight  \textbackslash{}
        0                                             6   
        1                                             6   
        2                                             6   
        3                                             6   
        4                                             6   
        5                                             6   
        6                                             6   
        7                                             6   
        8                                             6   
        9                                             6   
        10                                            6   
        11                                            6   
        12                                            6   
        13                                            6   
        14                                            6   
        15                                            6   
        16                                            6   
        17                                            6   
        18                                            6   
        19                                            6   
        20                                            6   
        21                                            6   
        22                                            6   
        
            Number experiencing thermal distress  Launch temperature (degrees F)  \textbackslash{}
        0                                      0                              66   
        1                                      1                              70   
        2                                      0                              69   
        3                                      0                              68   
        4                                      0                              67   
        5                                      0                              72   
        6                                      0                              73   
        7                                      0                              70   
        8                                      1                              57   
        9                                      1                              63   
        10                                     1                              70   
        11                                     0                              78   
        12                                     0                              67   
        13                                     2                              53   
        14                                     0                              67   
        15                                     0                              75   
        16                                     0                              70   
        17                                     0                              81   
        18                                     0                              76   
        19                                     0                              79   
        20                                     0                              75   
        21                                     0                              76   
        22                                     1                              58   
        
            Leak-check pressure (psi)  Temporal order of flight  
        0                          50                         1  
        1                          50                         2  
        2                          50                         3  
        3                          50                         4  
        4                          50                         5  
        5                          50                         6  
        6                         100                         7  
        7                         100                         8  
        8                         200                         9  
        9                         200                        10  
        10                        200                        11  
        11                        200                        12  
        12                        200                        13  
        13                        200                        14  
        14                        200                        15  
        15                        200                        16  
        16                        200                        17  
        17                        200                        18  
        18                        200                        19  
        19                        200                        20  
        20                        200                        21  
        21                        200                        22  
        22                        200                        23  
\end{Verbatim}
            
    \begin{Verbatim}[commandchars=\\\{\}]
{\color{incolor}In [{\color{incolor}6}]:} \PY{n}{data}\PY{o}{.}\PY{n}{drop}\PY{p}{(}\PY{p}{[}\PY{l+s+s2}{\PYZdq{}}\PY{l+s+s2}{Number of O\PYZhy{}rings at risk on a given flight}\PY{l+s+s2}{\PYZdq{}}\PY{p}{,} \PY{l+s+s2}{\PYZdq{}}\PY{l+s+s2}{Temporal order of flight}\PY{l+s+s2}{\PYZdq{}}\PY{p}{]}\PY{p}{,} \PY{n}{axis} \PY{o}{=} \PY{l+m+mi}{1}\PY{p}{,} \PY{n}{inplace} \PY{o}{=} \PY{k+kc}{True}\PY{p}{)}
\end{Verbatim}


    \hypertarget{reduced-data-set}{%
\subsection{Reduced Data Set:}\label{reduced-data-set}}

    \begin{Verbatim}[commandchars=\\\{\}]
{\color{incolor}In [{\color{incolor}7}]:} \PY{n}{data}
\end{Verbatim}


\begin{Verbatim}[commandchars=\\\{\}]
{\color{outcolor}Out[{\color{outcolor}7}]:}     Number experiencing thermal distress  Launch temperature (degrees F)  \textbackslash{}
        0                                      0                              66   
        1                                      1                              70   
        2                                      0                              69   
        3                                      0                              68   
        4                                      0                              67   
        5                                      0                              72   
        6                                      0                              73   
        7                                      0                              70   
        8                                      1                              57   
        9                                      1                              63   
        10                                     1                              70   
        11                                     0                              78   
        12                                     0                              67   
        13                                     2                              53   
        14                                     0                              67   
        15                                     0                              75   
        16                                     0                              70   
        17                                     0                              81   
        18                                     0                              76   
        19                                     0                              79   
        20                                     0                              75   
        21                                     0                              76   
        22                                     1                              58   
        
            Leak-check pressure (psi)  
        0                          50  
        1                          50  
        2                          50  
        3                          50  
        4                          50  
        5                          50  
        6                         100  
        7                         100  
        8                         200  
        9                         200  
        10                        200  
        11                        200  
        12                        200  
        13                        200  
        14                        200  
        15                        200  
        16                        200  
        17                        200  
        18                        200  
        19                        200  
        20                        200  
        21                        200  
        22                        200  
\end{Verbatim}
            
    \hypertarget{data-set-stats}{%
\subsection{Data Set Stats:}\label{data-set-stats}}

    \begin{Verbatim}[commandchars=\\\{\}]
{\color{incolor}In [{\color{incolor}8}]:} \PY{n}{data}\PY{o}{.}\PY{n}{describe}\PY{p}{(}\PY{p}{)}
\end{Verbatim}


\begin{Verbatim}[commandchars=\\\{\}]
{\color{outcolor}Out[{\color{outcolor}8}]:}        Number experiencing thermal distress  Launch temperature (degrees F)  \textbackslash{}
        count                             23.000000                       23.000000   
        mean                               0.304348                       69.565217   
        std                                0.558796                        7.057080   
        min                                0.000000                       53.000000   
        25\%                                0.000000                       67.000000   
        50\%                                0.000000                       70.000000   
        75\%                                0.500000                       75.000000   
        max                                2.000000                       81.000000   
        
               Leak-check pressure (psi)  
        count                  23.000000  
        mean                  152.173913  
        std                    68.221332  
        min                    50.000000  
        25\%                    75.000000  
        50\%                   200.000000  
        75\%                   200.000000  
        max                   200.000000  
\end{Verbatim}
            
    \hypertarget{thermal-distress-heat-map}{%
\subsection{Thermal Distress Heat
Map:}\label{thermal-distress-heat-map}}

    \begin{Verbatim}[commandchars=\\\{\}]
{\color{incolor}In [{\color{incolor}10}]:} \PY{n}{cor} \PY{o}{=} \PY{n}{data}\PY{o}{.}\PY{n}{corr}\PY{p}{(}\PY{p}{)}
         \PY{n}{sb}\PY{o}{.}\PY{n}{heatmap}\PY{p}{(}\PY{n}{cor}\PY{p}{,} \PY{n}{annot} \PY{o}{=} \PY{k+kc}{True}\PY{p}{,} \PY{n}{cmap} \PY{o}{=} \PY{l+s+s2}{\PYZdq{}}\PY{l+s+s2}{coolwarm}\PY{l+s+s2}{\PYZdq{}}\PY{p}{)}
\end{Verbatim}


\begin{Verbatim}[commandchars=\\\{\}]
{\color{outcolor}Out[{\color{outcolor}10}]:} <matplotlib.axes.\_subplots.AxesSubplot at 0x171ddade2b0>
\end{Verbatim}
            
    \begin{center}
    \adjustimage{max size={0.9\linewidth}{0.9\paperheight}}{output_8_1.png}
    \end{center}
    { \hspace*{\fill} \\}
    
    \hypertarget{distribution-of-thermal-distress-events}{%
\subsection{Distribution of Thermal Distress
Events}\label{distribution-of-thermal-distress-events}}

    \begin{Verbatim}[commandchars=\\\{\}]
{\color{incolor}In [{\color{incolor}11}]:} \PY{n}{plt}\PY{o}{.}\PY{n}{figure}\PY{p}{(}\PY{n}{figsize} \PY{o}{=} \PY{p}{(}\PY{l+m+mi}{12}\PY{p}{,}\PY{l+m+mi}{5}\PY{p}{)}\PY{p}{)}
         
         \PY{n}{sb}\PY{o}{.}\PY{n}{distplot}\PY{p}{(}\PY{n}{data}\PY{p}{[}\PY{l+s+s2}{\PYZdq{}}\PY{l+s+s2}{Launch temperature (degrees F)}\PY{l+s+s2}{\PYZdq{}}\PY{p}{]}\PY{p}{[}\PY{n}{data}\PY{p}{[}\PY{l+s+s2}{\PYZdq{}}\PY{l+s+s2}{Number experiencing thermal distress}\PY{l+s+s2}{\PYZdq{}}\PY{p}{]} \PY{o}{==} \PY{l+m+mi}{0}\PY{p}{]}\PY{p}{,} \PY{n}{label} \PY{o}{=} \PY{l+s+s2}{\PYZdq{}}\PY{l+s+s2}{No Thermal Distress}\PY{l+s+s2}{\PYZdq{}}\PY{p}{)}
         
         \PY{n}{sb}\PY{o}{.}\PY{n}{distplot}\PY{p}{(}\PY{n}{data}\PY{p}{[}\PY{l+s+s2}{\PYZdq{}}\PY{l+s+s2}{Launch temperature (degrees F)}\PY{l+s+s2}{\PYZdq{}}\PY{p}{]}\PY{p}{[}\PY{n}{data}\PY{p}{[}\PY{l+s+s2}{\PYZdq{}}\PY{l+s+s2}{Number experiencing thermal distress}\PY{l+s+s2}{\PYZdq{}}\PY{p}{]} \PY{o}{!=} \PY{l+m+mi}{0}\PY{p}{]}\PY{p}{,} \PY{n}{label} \PY{o}{=} \PY{l+s+s2}{\PYZdq{}}\PY{l+s+s2}{Thermal Distress Present}\PY{l+s+s2}{\PYZdq{}}\PY{p}{)}
         
         \PY{n}{plt}\PY{o}{.}\PY{n}{legend}\PY{p}{(}\PY{p}{)}
         \PY{n}{plt}\PY{o}{.}\PY{n}{show}\PY{p}{(}\PY{p}{)}
\end{Verbatim}


    \begin{Verbatim}[commandchars=\\\{\}]
c:\textbackslash{}program files\textbackslash{}python36\textbackslash{}lib\textbackslash{}site-packages\textbackslash{}matplotlib\textbackslash{}axes\textbackslash{}\_axes.py:6462: UserWarning: The 'normed' kwarg is deprecated, and has been replaced by the 'density' kwarg.
  warnings.warn("The 'normed' kwarg is deprecated, and has been "

    \end{Verbatim}

    \begin{center}
    \adjustimage{max size={0.9\linewidth}{0.9\paperheight}}{output_10_1.png}
    \end{center}
    { \hspace*{\fill} \\}
    
    \begin{Verbatim}[commandchars=\\\{\}]
{\color{incolor}In [{\color{incolor}12}]:} \PY{n}{xData} \PY{o}{=} \PY{n}{data}\PY{o}{.}\PY{n}{drop}\PY{p}{(}\PY{p}{[}\PY{l+s+s2}{\PYZdq{}}\PY{l+s+s2}{Number experiencing thermal distress}\PY{l+s+s2}{\PYZdq{}}\PY{p}{]}\PY{p}{,} \PY{n}{axis} \PY{o}{=} \PY{l+m+mi}{1}\PY{p}{)}
         \PY{n}{yData} \PY{o}{=} \PY{n}{data}\PY{p}{[}\PY{l+s+s2}{\PYZdq{}}\PY{l+s+s2}{Number experiencing thermal distress}\PY{l+s+s2}{\PYZdq{}}\PY{p}{]}
\end{Verbatim}


    \hypertarget{define-train-a-thermal-distress-model}{%
\subsection{Define \& Train a Thermal Distress
Model}\label{define-train-a-thermal-distress-model}}

    \begin{Verbatim}[commandchars=\\\{\}]
{\color{incolor}In [{\color{incolor}13}]:} \PY{c+c1}{\PYZsh{} Use sklearn and create a LinearRegression Object}
         \PY{n}{alg} \PY{o}{=} \PY{n}{LinearRegression}\PY{p}{(}\PY{p}{)}
\end{Verbatim}


    \begin{Verbatim}[commandchars=\\\{\}]
{\color{incolor}In [{\color{incolor}14}]:} \PY{c+c1}{\PYZsh{} Train the algorithm}
         \PY{n}{alg}\PY{o}{.}\PY{n}{fit}\PY{p}{(}\PY{n}{xData}\PY{p}{,} \PY{n}{yData}\PY{p}{)}
\end{Verbatim}


\begin{Verbatim}[commandchars=\\\{\}]
{\color{outcolor}Out[{\color{outcolor}14}]:} LinearRegression(copy\_X=True, fit\_intercept=True, n\_jobs=1, normalize=False)
\end{Verbatim}
            
    \hypertarget{predict-thermal-distress}{%
\subsection{Predict Thermal Distress}\label{predict-thermal-distress}}

At 31 degrees F the Linear Regression model predicts that
\textbf{\emph{2.546 O-Rings}} will experience Thermal Distress that may
lead to failure as was the case during the Space Shuttle Challenger
Mission when launch temperatures t were predicted at 31 degrees F.

    \begin{Verbatim}[commandchars=\\\{\}]
{\color{incolor}In [{\color{incolor}15}]:} \PY{c+c1}{\PYZsh{} Make prediction}
         \PY{n}{alg}\PY{o}{.}\PY{n}{predict}\PY{p}{(}\PY{n}{numpy}\PY{o}{.}\PY{n}{array}\PY{p}{(}\PY{p}{[}\PY{l+m+mi}{31}\PY{p}{,}\PY{l+m+mi}{150}\PY{p}{]}\PY{p}{)}\PY{o}{.}\PY{n}{reshape}\PY{p}{(}\PY{l+m+mi}{1}\PY{p}{,}\PY{l+m+mi}{2}\PY{p}{)}\PY{p}{)}
\end{Verbatim}


\begin{Verbatim}[commandchars=\\\{\}]
{\color{outcolor}Out[{\color{outcolor}15}]:} array([2.54622366])
\end{Verbatim}
            

    % Add a bibliography block to the postdoc
    
    
    
    \end{document}
